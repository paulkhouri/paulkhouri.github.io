\documentclass[a4paper,11pt]{article}
%this is the preamble
\usepackage{amsmath,graphicx,geometry,tikz,multicol,pgfplots,gensymb,comment,setspace,array,enumitem,tcolorbox,caption}
\usetikzlibrary{quotes,angles}
\usetikzlibrary{calc}

\pgfplotsset{width=2cm, compat=1.9}

\geometry{a4paper, top=2cm,bottom=2cm,right=1.5cm,left=1.5cm}

\usepackage[makeroom]{cancel}
\definecolor{shadecolor}{RGB}{180,180,180}

\newcolumntype{L}[1]{>{\raggedright\let\newline\\\arraybackslash\hspace{0pt}}m{#1}}
\newcolumntype{C}[1]{>{\centering\let\newline\\\arraybackslash\hspace{0pt}}m{#1}}
\newcolumntype{R}[1]{>{\raggedleft\let\newline\\\arraybackslash\hspace{0pt}}m{#1}}
\renewcommand*{\thefootnote}{\fnsymbol{footnote}}
\newcommand*\circled[1]{\tikz[baseline=(char.base)]{
		\node[shape=circle,draw,inner sep=2pt] (char) {#1};}}
\renewcommand{\arraystretch}{2}
\setlength\parindent{0pt}
\begin{document}
			\Large
			\begin{figure} 
				\centering
				\includegraphics[height=4cm]{Marsden_green_on_white.jpg}
				\caption*{May 2017}
			\end{figure}
			.
			\vspace{3cm}
			\begin{center}
				\textbf{Year 12 Systems of Equations Course Notes}
			\end{center}
			
			\vspace{3cm}
			
			\normalsize
			\newpage
\textbf{Year 12 Systems of Equations 2.14 Course Notes}
\tableofcontents
\newpage	
\Large \textbf{2.14 Systems of Equations}\normalsize
\section{Introduction}	
A system of equations is a group of two or more equations, where the equations have the same variables.\\
This standard focuses on systems with two equations where at least one is non linear.\\
Examples of equation systems:\\\\
1. Linear simultaneous equations:
\begin{align*}
&y=x-4&&&\\
&y+2x=14&&&
\end{align*}
2. Non-linear simultaneous equations:
\begin{align*}
&y=x^{2}-2x-4&&&\\
&y=2x-4&&&
\end{align*}
3. Non-linear simultaneous equations:
\begin{align*}
&y=x+1&&&\\
&xy=6&&&
\end{align*}
\section{Linear Simultaneous Equations Revision}
Solve the following simultaneous linear equations using the substitution method.\\
\begin{align*}
y&=x-3&\circled{1}&&&&\\
y+2x&=15&\circled{2}&&&&
\end{align*}
We use \circled{1} to substitute for $y$ in equation \circled{2}
\begin{align*}
y+2x&=15&&&\\
x-3+2x&=15&&&\\
3x&=18&&&\\
x&=6&&&
\end{align*}
Substitute $x$ into either equation (in this case we will choose \circled{1})
\begin{align*}
y&=x-3&&&\\
y&=6-3&&&\\
y&=3&&&\\
\end{align*}
Page 12 Nulake book
\newpage
\section{Linear Simultaneous Equations Revision with word problems}
\begin{enumerate}
\item A hamburger costs 3 times the cost of a bag of chips. 
I buy 10 hamburgers and 4 bags of chips and it costs me \$35.70. How much does a hamburger cost?
\item A group is to visit Happy Valley Theme Park by bus. The bus will hold 50 people. 
Adults are charged \$8 each for the trip and children \$2.50 each. The bus is full and \$196.50 is collected in total from the passengers.   Find the number of children on the bus.
\item Tickets to a play cost \$38 for adults and \$12 for students. 
500 tickets are sold for \$18012. 
How many adults and how many students attended?
\item Maddie buys 2 kg of flour and 1 dozen eggs for \$9.50. 
Later she buys a kilogram of flour and 2 dozen eggs for \$8.80.
Find the cost of a kilogram of flour.
\item Lucy bought mini pizzas for her family one night. 
The Supreme Pizza cost 50 cents more than the Hawaiian. 
The total cost for 2 Hawaiian pizzas and 3 Supreme pizzas was \$20.75. 
What is the price of one Supreme pizza?
\item A tennis club has 78 members – male and female. The number of males is 6 more than twice the number of females. How many males were there?
\end{enumerate}
\newpage
\section{Non-Linear Systems I}
Consider the following system of equations:\\
\begin{align*}
y&=2x+4&\circled{1}&&&&\\
y&=x^{2}-3x-10&\circled{2}&&&&
\end{align*}
These can be solved by substitution, rearrangement and factorising.\\\\
Substitute \circled{1} into \circled{2}
\begin{align*}
y&=2x+4&&&&&\\
4&=x^{2}-5x-10&&&&&\\
0&=x^{2}-5x-14&&&&&\\
\Rightarrow ~~~x^{2}-5x-14&=0&&&&&\\
\text{Factorise} ~~~(x-7)(x+2)&=0&&&&&\\
\Rightarrow ~~~x=7~and~x=-2&&&&&&
\end{align*}
\begin{itemize}
\item Substituting $x=7$ into \circled{1}, we find
\begin{align*}
y&=2\times 7+4&&&&&\\
y&=18&&&&&
\end{align*}
\item Substituting $x=-2$ into \circled{1}, we find
\begin{align*}
y&=2\times -2+4&&&&&\\
y&=0&&&&&
\end{align*}
\end{itemize}
If we plot both equations, the solutions give the intersection points.
\begin{center}
\begin{tikzpicture}[yscale=0.2,xscale=0.3]
\draw[->] (-8,0) -- (8,0) node[right] {$x$};
\draw[->] (0,-15) -- (0,30) node[above] {$y$};
\draw[scale=1,domain=-5:9,smooth,variable=\x,blue] plot ({\x},{2*\x+4});
\draw[scale=1,domain=-5:8,smooth,variable=\x,red] plot ({\x},{\x*\x-3*\x-10});
\fill (-2,0) ellipse (6pt and 9pt);
\fill (7,18) ellipse (6pt and 9pt);
\draw (-2,0.75) node[left=0pt] {$(-2,0)$};
\draw (7,18) node[right=0pt] {$(7,18)$};
\draw (4,-3) node[right=4pt,red] {$y = -x^2 + 4x + 2$};
\draw (-5,-3) node[left,blue] {$y = 2x+4$};
	\end{tikzpicture}
\end{center}
\newpage
\subsection{Exercise:}\label{ex1}
Solve these by factorising the final quadratic 
\begin{multicols}{3}
\begin{enumerate}
\item \begin{align*}
y&=4x+1\\
y&=x^{2}-x-65
\end{align*}
\item \begin{align*}
y&=10x+5\\
y&=x^{2}+22x+25
\end{align*}
\item \begin{align*}
y&=x-1\\
y&=x^{2}-11x+34
\end{align*}
\item \begin{align*}
y&=x+20\\
y&=x^{2}
\end{align*}
\item \begin{align*}
y&=3x+2\\
y&=x^{2}+8x-22
\end{align*}
\item \begin{align*}
y&=4x+2\\
y&=2x^{2}-7x-19
\end{align*}
\end{enumerate}	
\end{multicols}
\subsection{Exercise:}\label{ex2}
Solve these by using the polynomial solver on the graphics calculator for the final quadratic.
\begin{multicols}{3}
\begin{enumerate}
\item \begin{align*}
y-x&=10\\
x^{2}+y&=8020
\end{align*}
\item \begin{align*}
y-x&=10\\
x^{2}+y&=8200
\end{align*}
\item \begin{align*}
x+y&=25\\
x^{2}+y&=5137
\end{align*}
\item \begin{align*}
2x+y&=50\\
x^{2}+y&=5137
\end{align*}
\item \begin{align*}
y-12x&=101\\
x^{2}+y&=5541
\end{align*}
\item \begin{align*}
y+x&=500\\
x^{2}+y&=10006
\end{align*}
\end{enumerate}	
\end{multicols}
\subsection{Exercise:}\label{ex3}
Solve the following systems:
\begin{multicols}{2}
\begin{enumerate}
\item \begin{align*}
x^{2}+y^{2}&=5\\
y&=x-3
\end{align*}
\item \begin{align*}
x^{2}+y^{2}&=18\\
y&=x+6
\end{align*}
\end{enumerate}	
\end{multicols}


\newpage
\section{Non-Linear Systems II}
Consider the following system of equations:\\
\begin{align*}
y&=x+4&\circled{1}&&&&\\
xy&=45&\circled{2}&&&&
\end{align*}
substitute \circled{1} into \circled{2}
\begin{align*}
\Rightarrow~~~ x(x+4)&=45&&&&&\\
x^{2}+4x-45&=0&&&&&\\
(x+9)(x-5)&=0&&&&&\\
\Rightarrow~~~ x=-9~~and~~x=5&&&&&&\\
\Rightarrow~~~ y=-9+4~~and~~y=5+4&&&&&&\\
y=-5~~and~~y=9&&&&&&
\end{align*}\\
\textbf{Using the solver on the graphics calculator:}\\
\begin{align*}
y&=63x+15&\circled{1}&&&&\\
xy&=12&\circled{2}&&&&
\end{align*}
substitute \circled{1} into \circled{2}
\begin{align*}
x(63x+15)&=12&&&&&\\
\underset{a}{\underbrace{63}}~x^{2}~\underset{b}{\underbrace{+15}}~x~\underset{c}{\underbrace{-12}}&=0&&&&&
\end{align*}
On your Calculator:Go to EQUA $\rightarrow$ F2 (Polynomial)  $\rightarrow$ F2 (Degree 2)\\ $\rightarrow$ Enter $a=63$, $b=15$, $c=-12$\\\\
This gives $x=0.333$ and $x=-0.571$\\\\
Finding $y$ using \circled{1}
\begin{align*}
y&=63\times 0.333+15&&&&&&\\
&=35.979&&&&&&\\
y&=63\times -0.571+15&&&&&&\\
&=20.973&&&&&&
\end{align*}
\newpage
\subsection{Exercise:}\label{ex4}
Solve by factorising:
\begin{multicols}{3}
\begin{enumerate}
\item \begin{align*}
y&=x+9\\
xy&=36
\end{align*}
\item \begin{align*}
y&=x-13\\
xy&=-36
\end{align*}
\item \begin{align*}
xy&=3\\
y&=x-2
\end{align*}
\end{enumerate}	
\end{multicols}
\subsection{Exercise:}\label{ex5}
Solve using polnomial solver:
\begin{multicols}{3}
\begin{enumerate}
\item \begin{align*}
y&=8x+2\\
xy&=3
\end{align*}
\item \begin{align*}
y&=95x+174\\
xy&=77
\end{align*}
\item \begin{align*}
xy&=5656\\
y&=1564x+3453
\end{align*}
\end{enumerate}	
\end{multicols}



\newpage
\section{First Practice Tasks}
\begin{figure}[!ht]
	\centering
	\includegraphics[width=16cm]{prac1}
\end{figure}
\newpage
\begin{figure}[!ht]
	\centering
	\includegraphics[width=16cm]{prac2}
\end{figure}
\newpage
\section{Further Substitution Problems}
Example: Consider the following system of equations:
\begin{align*}
5y+2x&=2&\circled{1}&&&&&\\
2xy&=-6+9y&\circled{2}&&&&&
\end{align*}
Rearranging \circled{1}, it looks like $2x=2-5y$ and looks like a good choice for substitution in \circled{2}.
\begin{align*}
(2-5y)y=-6+9y
\end{align*}
Expanding and rearranging leads to a quadratic in the variable $y$:
\begin{align*}
(2-5y)y&=-6+9y\\
2y-5y^{2}&=-6+9y\\
5y^{2}+7y-6&=0\\
(5y-3)(y+2)&=0
\end{align*}
$\Rightarrow$
\begin{align*}
y=\frac{3}{5}~~~\text{or}~~~y=-2
\end{align*}
Substituting these values into \circled{1}, we find the corresponding $x$ value:\\
\begin{align*}
5\times \frac{3}{5}+2x=2\\
x=-\frac{1}{2}
\end{align*}
and 
\begin{align*}
5\times-2+2x=2\\
x=6
\end{align*}
The solutions are, therefore:
\begin{align*}
x=-\frac{1}{2}~~~~&\text{and}~~~~y=\frac{3}{5}\\
x=6~~~~&\text{and}~~~~y=-2
\end{align*}
\newpage
\subsection{Exercise}\label{ex6}
\begin{multicols}{2}
\begin{enumerate}
\item\begin{align*}
10y+2x&=100\\
2xy&=15000-750y 
\end{align*}\footnotesize(choose the solution where $x+y$ is largest)\normalsize\\
\item\begin{align*}
2x+12y&=840\\
2xy&=18000-120y
\end{align*}\footnotesize(choose the solution where $x+y$ is smallest) \normalsize
\item\begin{align*}
4y+3x&=286\\
3xy&=11760-150y
\end{align*}\footnotesize (choose the solution where $x+y$ is largest) \normalsize
\item\begin{align*}
56y+7x&=700\\
7xy&=9800-1092y
\end{align*}\footnotesize(choose the solution where $x,y$ are both positive) \normalsize
\item\begin{align*}
18y+4x&=721\\
4xy&=5760-143y
\end{align*}\footnotesize (choose the solution where $x+y$ is largest) \normalsize
\item\begin{align*}
24y+6x&=532\\
6xy&=9360-500y
\end{align*}\footnotesize (choose the solution where $x,y$ are both positive)\normalsize
\item\begin{align*}
35y+2x&=1085\\
2xy&=21350-1400y
\end{align*}\footnotesize (choose the solution where $x,y$ are both positive) \normalsize
\item\begin{align*}
7x+4y&=302\\
21120-302y&=7xy
\end{align*}\footnotesize (choose the solution where $x,y$ are both positive)\normalsize
\end{enumerate}
\end{multicols}
\subsection{Exercise}\label{ex7}
More complex substitutions. Choose only whole number solutions for $x$ and $y$.
\begin{multicols}{3}
\begin{enumerate}
\item\begin{align*}
3x+4y&=740\\
xy&=5500+50y
\end{align*} 
\item\begin{align*}
7x+9y&=1602\\
xy&=8282+17y
\end{align*}
\item\begin{align*}
5x+11y&=3270\\
xy&=33490+83y
\end{align*}
\item\begin{align*}
7x+5y&=2195\\
xy&=4020+150y
\end{align*}
\item\begin{align*}
2x+5y&=1702\\
xy&=3300+90y
\end{align*}
\item\begin{align*}
17x+14y&=2972\\
xy&=2060+70y
\end{align*}
\item\begin{align*}
11x+13y&=9320\\
xy&=40000-200y
\end{align*} 
\item\begin{align*}
4x-5y+10&=0\\
xy&=2850-30y
\end{align*}
\item\begin{align*}
3x-7y+202&=0\\
4xy&=2679+7x
\end{align*}     
\end{enumerate}
\end{multicols}
\newpage
\subsection{Exercise} \label{ex8}
\begin{enumerate}
\item Sports incorporated sells basketballs and cricketballs for \$56.00 and \$7.00 repectively.\\
It so happens that the relationship between the number of basketballs(x) and cricketballs(y) sold on any day can be modelled by the equation $7xy=9800-1092y$.\\
\$700 worth of basketballs and cricketballs are sold on a particular day.\\
How many of each were sold?
\item Sports Inc also sells netballs and tennis balls. Their cost is \$35.00 and \$2.00 repectively.\\
It so happens that the relationship between the sales of netballs and tennis balls can be modelled by the equation $2xy=21350-1400y$.\\
Total sales were \$1085 on a particular day.\\
How many of each were sold?\\
(you may get a fractional answer that must be rounded, and may also find that many people had run out of tennis balls that day).
	
\end{enumerate}
\newpage
\section{Using the discriminant with k problems}
Question: Find value(s) of k that will give one solution to the following equation system.
\begin{align*}
2x+ky&=5&\circled{1}&&&&&\\
4xy&=1+6y&\circled{2}&&&&&
\end{align*}
In this case we proceed as usual by creating an equation that has one $x$ or $y$ variable (and will also include $k$);\\
Rearrange \circled{1}, multiply by $2$ and substitute:
\begin{align*}
&&2x&=5-ky&\circled{1}&&&&&\\
(\times 2)\Rightarrow&& 4x&=10-2ky&&&&&&\\
\text{substitute into}~\circled{2}\Rightarrow&& (10-2ky)y&=1+6y&&&&&&\\
&& 10y-2ky^{2}&=1+6y&&&&&&\\
&& 2ky^{2}-4y+1&=0&\circled{3}&&&&&\\
\end{align*}
At this point we need to find the value of $k$ that will give the equation one solution:\\
This means that we need to use the discriminant, finding value for $k$ such that $b^{2}-4ac=0$\\\\
We identify from \circled{3} that $a=2k$, $b=-4$, $c=+1$.
\begin{align*}
&&b^{2}-4ac&=0&&&&&&\\
\Rightarrow&&(-4)^{2}-4\times 2k\times 1&=0&&&&&&\\
&&16-8k&=0&&&&&&\\
&&k&=2&&&&&&
\end{align*}
\subsection{Exercise: Find values of k that give one solution to the system}\label{ex9}
\begin{multicols}{3}
\begin{enumerate}
\item \begin{align*}
20x+ky&=100\\
2xy&=1000-10y
\end{align*}
\item \begin{align*}
2x+ky&=1500\\
4xy&=1100-5y
\end{align*}
\item \begin{align*}
7x+5y&=k\\
14xy&=2010-20y
\end{align*}
\item \begin{align*}
12x+24y&=48\\
kxy&=2010-12y
\end{align*}
\item \begin{align*}
15x+12y&=120\\
3xy&=k-4y
\end{align*}
\item \begin{align*}
y&=(x-3)^{2}\\
y&=-(x-k)^{2}+5
\end{align*}
\end{enumerate}
\end{multicols} 
\newpage
\section{Answers}
\begin{multicols}{3}
\textbf{\ref{ex1}:}
\begin{enumerate}
\item $x=11$~$y=45$\\$x=-6$~$y=-23$
\item $x=-10$~$y=-95$\\$x=-2$~$y=-15$
\item $x=7$~$y=6$\\$x=5$~$y=4$
\item $x=5$~$y=25$\\$x=-4$~$y=-16$
\item $x=-8$~$y=-22$\\$x=3$~$y=11$
\item $x=-\frac{3}{2}$~$y=-4$\\$x=7$~$y=30$
\end{enumerate}
\textbf{\ref{ex2}:}
\begin{enumerate}
\item $x=89$~$y=99$\\$x=-90$~$y=-80$
\item $x=90$~$y=100$\\$x=-91$~$y=-81$
\item $x=72$~$y=-47$\\$x=-71$~$y=96$
\item $x=72.33$~$y=-94.66$\\$x=-70.33$~$y=190.66$
\item $x=68$~$y=917$\\$x=-80$~$y=-859$
\item $x=98$~$y=402$\\$x=-97$~$y=597$
\end{enumerate}
\textbf{\ref{ex3}:}
\begin{enumerate}
\item $x=2$~$y=1$\\$x=1$~$y=-2$
\item $x=3$~$y=-3$\\\\
\end{enumerate}
\textbf{\ref{ex4}:}
\begin{enumerate}
\item $x=-12$~$y=-3$\\$x=3$~$y=12$
\item $x=9$~$y=-4$\\$x=4$~$y=-9$
\item $x=-1$~$y=-3$\\$x=3$~$y=1$
\end{enumerate}
\textbf{\ref{ex5}:}
\begin{enumerate}
\item $x=\frac{1}{2}$~$y=6$\\$x=-\frac{3}{4}$~$y=-4$
\item $x=\frac{7}{19}$~$y=209$\\$x=-\frac{11}{5}$~$y=-35$
\item $x=\frac{101}{92}$~$y=5170$\\$x=-\frac{56}{17}$~$y=-1699$
\end{enumerate}
\textbf{\ref{ex6}:}	
\begin{enumerate}
\item $x=-75$~$y=25$\\$x=-250$~$y=60$
\item $x=240$~$y=30$\\$x=120$~$y=50$
\item $x=30$~$y=49$\\$x=15.\dot{3}$~$y=60$
\item $x=44$~$y=7$\\$x=-100$~$y=25$
\item $x=0.25$~$y=40$\\$x=144.25$~$y=8$
\item $x=36\frac{2}{3}$~$y=13$\\$x=-30\frac{1}{3}$~$y=30$
\item $x=367.5$~$y=10$\\$x=-525$~$y=61$
\item $x=\frac{82}{7}$~$y=55$\\$x=-\frac{82}{7}$~$y=96$
\end{enumerate}
\textbf{\ref{ex7}:}
\begin{enumerate}
\item $y=110$~$y=37.5$\\$x=100$
\item $y=101$~$y=63.777$\\$x=99$
\item $y=170$~$y=89.545$\\$x=280$
\item $y=201$~$y=28$\\$x=170$
\item $y=300$~$y=4.4$\\$x=101$
\item $y=103$~$y=24.85$\\$x=90$
\item $y=40$~$y=846.15$\\$x=800$
\item $y=38$~$y=-60$\\$x=45$~$x=-310$
\item $x=19$~$y=-82.25$\\$y=37$
\end{enumerate}
\textbf{\ref{ex9}:}
\begin{enumerate}
\item $k=1$
\item $k=1026.14$
\item $k=131.77$~or~$k=-151.7$
\item $k=28.701$~or~$k=-34.7$
\item $k=81.67$
\item $k=-0.162$~or~$k=6.162$
\end{enumerate}
\end{multicols}


\newpage
\section{Second Practice Tasks}
\begin{figure}[!ht]
	\centering
	\includegraphics[width=17cm]{prac3}
\end{figure}
\newpage
\begin{figure}[!ht]
	\centering
	\includegraphics[width=17cm]{prac4}
\end{figure}
\newpage
\begin{figure}[!ht]
	\centering
	\includegraphics[width=17cm]{prac5}
\end{figure}
\newpage
\begin{figure}[!ht]
	\centering
	\includegraphics[width=17cm]{prac6}
\end{figure}
\newpage
\section{Extra Questions}
\begin{enumerate}
\begin{multicols}{2}
\item \begin{enumerate}
\item \begin{align*}
x^2+100y&=10000\\
4x+5y&=560
\end{align*}	
\item \begin{align*}
x^2+125y&=7400\\
	4x+5y&=360
	\end{align*}
\item\begin{align*}
x^2+100y&=6225\\
12x+15y&=1080
\end{align*}
\item\begin{align*}
x^2+52y&=4482\\
6x+8y&=720
\end{align*}
\item\begin{align*}
x^2+169y&=8281\\
5x+13y&=689
\end{align*}
\item\begin{align*}
x^2+9y&=1305\\
80x+15y&=2895
\end{align*}	
\end{enumerate}
\end{multicols}	
\vspace{1cm}
\begin{multicols}{2}
\item \begin{enumerate}
\item \begin{align*}
10xy&=69255-455y\\
8x+12y&=1292
\end{align*}
\item \begin{align*}
5xy&=360y-3150\\
7x+9y&=315
\end{align*}
\item \begin{align*}
42xy&=700y+5740\\
21x+35y&=1855
\end{align*}
\item \begin{align*}
6xy&=327y-9801\\
14x+21y&=2611
\end{align*}	
\end{enumerate}
\end{multicols}
\vspace{1cm}
\item Find value(s) for k that gives 1 solution:
\begin{multicols}{3}
\begin{enumerate}
\item \begin{align*}
x^2+125y&=10000\\
kx+5y&=625
\end{align*}
\item \begin{align*}
3xy&=6561-36y\\
3x+ky&=126
\end{align*}
\item \begin{align*}
y^2+48x&=1872\\
2x+y&=k
\end{align*}
	
\end{enumerate}
\end{multicols}
\end{enumerate}
\newpage
\subsection{Solutions}
\begin{enumerate}
\begin{multicols}{2}
\item \begin{enumerate}
\item \begin{align*}
(20,96)\\
(60,64)
\end{align*}	
\item \begin{align*}
(20,56)\\
(80,8)
\end{align*}
\item\begin{align*}
(15,60)\\
(65,20)
\end{align*}
\item\begin{align*}
(6,85.5)\\
(33,65.25)
\end{align*}
\item\begin{align*}
(13,48)\\
(52,33)
\end{align*}
\item\begin{align*}
(12,129)\\
(36,1)
\end{align*}	
\end{enumerate}
\end{multicols}	
\vspace{1cm}
\begin{multicols}{2}
\item \begin{enumerate}
\item \begin{align*}
(40,81)\\
(76,57)
\end{align*}
\item \begin{align*}
(27,14)\\
(90,-35)
\end{align*}
\item \begin{align*}
(20,41)\\
(85,2)
\end{align*}
\item \begin{align*}
(38,99)\\
(203,-11)
\end{align*}	
\end{enumerate}
\end{multicols}
\vspace{1cm}
\item 
\begin{multicols}{3}
\begin{enumerate}
\item \begin{align*}
k=6
\end{align*}
\item \begin{align*}
k=1
\end{align*}
\item \begin{align*}
k=84
\end{align*}

\end{enumerate}
\end{multicols}
\end{enumerate}
\end{document}